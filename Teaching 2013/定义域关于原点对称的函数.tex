\documentclass{article}
\usepackage{hyperref}
\usepackage{mathtools}
\usepackage{CJKutf8}
\usepackage{amssymb}

\begin{document}
\begin{CJK}{UTF8}{gkai}

\title{定义域关于原点对称的函数}
\date{}
\author{}
\maketitle

\section{先扯扯呗}

第一次接触到如下命题: \begin{quote} 
函数$f(x)$的定义域关于原点对称, 则函数$f(x)$能被唯一分解成一奇函数$g(x)$与一偶函数$h(x)$的和.
\end{quote}应该是在高中的习题, 当时就是形式化地写了类下述的证明. 

\section{证明}
$g(x)$是奇函数, 因而
\begin{equation}\label{eq1}g(x) = -g(-x)\end{equation}
$h(x)$是偶函数, 因而
\begin{equation}\label{eq2}h(x) = h(-x)\end{equation}
根据题意,
\begin{equation}\label{eq3}f(x) = g(x) + h(x)\end{equation}
将\eqref{eq3}中$x$替换为$-x$, 得
\begin{equation}\label{eq4}f(-x) = g(-x) + h(-x)\end{equation}
将\eqref{eq1}, \eqref{eq2}代入\eqref{eq4}, 得
\begin{equation}\label{eq5}f(-x) = -g(x) + h(x)\end{equation}
将\eqref{eq3}, \eqref{eq5}联立组成关于$g(x)$, $h(x)$的二元方程方程组, 

\begin{equation}
\begin{cases}
f(x) = g(x) + h(x)\\
f(-x) = -g(x) + h(x)
\end{cases}
\end{equation}

解得,
\begin{equation}
\begin{cases}\label{eq7}
g(x) = \frac{f(x) - f(-x)}{2}\\
h(x) = \frac{f(x) + f(-x)}{2}\end{cases}
\end{equation}
至于唯一性, 表达式只有这一种方式自然就说明分解形式唯一. 如果不放心, 可以搞个反证法, 来双保险一下. 不过我高中老师说, 他听到"双保险"会想笑, 有些人看到一元二次方程二次项和常数项系数符号相反, 还要用$\Delta$检验双保险一下.
\section{感谢你听我扯到这里}
通过上述证明, 我们能够看到任何一个定义域关于原点对称的函数$f(x)$都能被唯一分解成一奇函数$g(x)$与一偶函数$h(x)$的和. 不妨称$g(x)$, $h(x)$为$f(x)$的奇部, 偶部.\\
我们来举几个特殊函数, 计算下它们的奇偶部. 
\subsection{一次函数$$y = ax + b(a,b \in \mathbb{R}; a \not = 0)$$}
\begin{equation}
\begin{cases}
g(x) = ax\\
h(x) = b
\end{cases}
\end{equation}

\subsection{二次函数$$y = ax^2 + bx + c (a, b ,c \in \mathbb{R}; a, b \not = 0)$$}
\begin{equation}
\begin{cases}
g(x) = bx\\
h(x) = ax^2 +  c
\end{cases}
\end{equation}

\subsection{一般多项式函数$$y = a_nx^n + a_{n-1}x^{n-1} + \ldots + a_1x + a_0 = \sum_{i=0}^{n}{x_i}(a_i \in \mathbb{R}; 0 \le i \le n; i \in \mathbb{N})$$}
$g(x)$为奇次项之和;\\
$h(x)$ 为偶次项之和.


\subsection{搞个一般函数瞧一瞧}
$$f(x)=ln((x+1)(x-1)^3)$$
\begin{equation}
\begin{cases}
g(x) = \frac{ln((x+1)(x-1)^3)-ln((-x+1)(-x-1)^3)}{2}\\
h(x) = \frac{ln((x+1)(x-1)^3)+ln((-x+1)(-x-1)^3)}{2}
\end{cases}
\end{equation}
\includegraphics[width=12cm]{fig.png}\\
黑色的函数图像为$f(x)$, $g(x), h(x)$可据奇偶函数分辨出来. 

\section{后记}
我就是觉得这个很神奇啊, 拿来一个定义域关于原点对称的函数就能这样弄啊..完成这篇文章的动力在于, 我再也不想看到多个文件散落在我的桌面上了.



\end{CJK}
\end{document}

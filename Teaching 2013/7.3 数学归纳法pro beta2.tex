\documentclass{article}
\usepackage{hyperref}
\usepackage{mathtools}
\usepackage{CJKutf8}
\usepackage{amssymb}
\usepackage{geometry}
\usepackage{enumerate}
\usepackage{multicol} 

\geometry{a4paper} 

\begin{document}
\begin{CJK}{UTF8}{gkai}

\title{数学归纳法}
\date{}
\author{张舒悦}
\maketitle
       
\section{教学目标}
\begin{enumerate}
\item 知道数学归纳法的基本原理.
\item 理解数学归纳法证题的两个步骤.
\item 初步应用数学归纳法证明一些简单的与正整数有关的恒等式.
\end{enumerate}

\section{教学重点}
\begin{itemize}
\item 初步应用数学归纳法证明一些简单的与正整数有关的恒等式.
\end{itemize}

\section{教学难点}
\begin{itemize}
\item  理解数学归纳法证题的两个步骤.
\end{itemize}

\section{教学过程}
\subsection{引入}
\subsubsection{生活实例: 摸球实验}
*提问: 从一个装着一些球的袋子里第一次摸出的是白球, 第二, 三, 四次也是白球. 能否断定袋子里面都是白球? 为什么? \\
从特殊到一般的推理方法, 叫做归纳法. 而上面提及的例子中是基于部分事实所归纳出来的, 这是不可靠的.这是一种不完全归纳法. \\
*提问: 如何断定袋子里面都是白球?\\
这是一种完全归纳法.\\
*提问: 如果说袋子里面有无穷无尽的的球, 又该如何断定袋子里面都是白球?\\
有这样一个保证: 其一是第一次摸出的是白球, 其二是如果从第一次开始的每一次(包括第一次)摸出的是白球, 那么紧接着的下一次摸出的也一定白球. 这也就是说, 第一次摸出白球, 那么下一次, 也就是第二次摸出的也是白球. 第二次摸出白球, 那么下一次, 也就是第三次摸出的也是白球. \\
*提问: 第一次摸出白球这是我保证的. 那么第二次摸出的是白球是如何得出的?\\
第二次摸出白球是由第一次摸出白球所保证的. 第三次是由第二次所保证的$\cdots$那么第$k+1$次是由第$k$次所保证的, $k$是从1开始的自然数.\\
*提问: 这样是否对于每一次摸出的是白球都检验了?\\
*提问: 能否断定袋子里面都是白球?\\
*提问: 得出这个结论凭借了哪两个条件?\\
其一是第一个满足结论, 其二是如果任何一个满足结论, 那么其后紧接着的一个也满足结论. 实际上, 我们对于每一次摸出白球都进行了验证.\\
那么来看一下数学上的情形. 我们可以如何来验证一个与正整数$n$有关的数学命题, 尤其是$n$为任意的正整数的情形? 比如验证$$1^2 + 2^2 + 3^2 + \cdots + n^2 = \frac{n(n + 1)(2n + 1)}{6}.\hspace{3mm}(n \in \mathbb{N}^*)$$的正确性.\\
*讨论: (两分钟)两人为组讨论一下, 给出一个大概的想法.\\
回顾白球实验, 根据第$k$次情况成立推导第$k+1$次情况成立.\\
满足初始条件, 并且有递推的保证, 递推从初始值开始.\\

\subsubsection{定义}
这是数学家通过对正整数的深入研究, 找到的\\
一种证明与正整数$n$有关的数学命题的简单有效的方法, 其关键步骤如下:
\begin{enumerate}[i]
\item 证明当$n$取第一个值$n_0$时结论正确;
\item 假设当$n = k (k \in \mathbb{N}^*, k \geq n_0)$ 时结论正确, 证明当$n = k + 1$时结论也正确.
\end{enumerate}
完成这两个步骤后, 就可以断定命题对从$n_0$开始的所有正整数$n$成立.\\这种证明方法叫做数学归纳法(mathematical induction).\\
*提问: 为什么这个能断定命题对从$n_0$开始的所有正整数$n$成立? 是如何保证的?\\
$k$的范围需注意.\\
*提问: $N_0$是多少? 一定是1吗?\\
摸球的条件更改为从保证第五次摸出白球, 第二个保证不变.\\
$N_0$可以是任何一个自然数.\\


\subsection{例题1}
用数学归纳法证明
$$1^2 + 2^2 + 3^2 + \cdots + n^2 = \frac{n(n + 1)(2n + 1)}{6}.\hspace{3mm}(n \in \mathbb{N}^*)$$
解: \\
*提示: 请大家回忆数学归纳法证明的两个步骤.\\
\begin{enumerate}[i]
\item 当$n = 1$时, 左边$= 1^2 = 1$, 右边$= \frac{1 \times (1 + 1) \times (2 \times + 1)}{6} = 1$.\\左边 = 右边, 等式成立.
\item 假设当$n = k (k \in \mathbb{N}^*, k \geq 1)$时, 等式成立, 即$1^2 + 2^2 + 3^2 + \cdots + k^2 = \frac{k(k + 1)(2k + 1)}{6}.$
\\那么当$n = k + 1$时, \\$1^2 + 2^2 + 3^2 + \cdots + k^2 + (k + 1)^2\\= \frac{k(k + 1)(2k + 1)}{6} + (k+1)^2\\= \frac{(k + 1)(2k^2 + k + 6k +6)}{6}\\= \frac{(k + 1)(k + 2)(2k + 3)}{6}\\= \frac{(k + 1)[(k + 1) + 1][2(k + 1) + 1]}{6}$, \\等式也成立.
\end{enumerate}
思考k的范围.\\
我们可以先看一下我们要证的结论的形式应该是什么样子的; 将写$k + 1$替换$n$后的式子写在旁边. 变形的时候, 提示提取公因式$k + 1$.\\
为了靠近结论, 应注意下标的对应.\\
*提问: 是否完成了整个证明过程?\\
$\therefore$根据1, 2可以断定, $1^2 + 2^2 + 3^2 + \cdots + n^2 = \frac{n(n + 1)(2n + 1)}{6}$对任何$n \in \mathbb{N}^*$都成立.

\subsection{例题2}
数列$\{a_n\}$是等差数列, 已知$a_1, d$, 证明其通项公式为$a_n = a_1 + d(n-1).\hspace{3mm}(n \in \mathbb{N}^*)$ \\
今天我们用数学归纳法来证明这个结论. 请问该如何做? \\
*讨论: (两分钟)两人为组讨论一下.\\
解: \\
\begin{enumerate}[i]
\item 当$n = 1$时, 左边$= a_1$, 右边$= a_1 + d \times (1-1) = a_1$.\\左边 = 右边, 等式成立.\\
*提问: 数学归纳法中给出的第二个步骤是关于$k$和$k + 1$的. 那么对于数列呢?\\
数列的项和其后一项.\\
*提问: 请同学们思考下什么公式是表达着两者的关系的?  \\
写出等差数列的递推公式.\\

\item 假设当$n = k (k \in \mathbb{N}^*, k \ge 1)$时, 等式成立, 即$a_k = a_1 + d(k-1).$\\那么当$n = k + 1$时, $a_{k+1} = a_k + d = a_1 + d(k-1) + d = a_1 +d[(k + 1) - 1], $等式也成立.\\
*提问: k的范围应该是什么?\\
\end{enumerate}
$\therefore$根据i, ii可以断定, 通项公式$a_n = a_1 + d(n-1)$对任何$n \in \mathbb{N}^*$都成立.\\
书写结论, 由两个条件推出结论.\\

\subsection{小结}
\begin{itemize}
\item 适用范围: 证明某些与正整数$n$有关的论断的正确性.
\item 已知条件与步骤:
\begin{itemize}
\item 递推的起点: 初始条件 + 猜想.
\item 递推的依据: 递推关系 + 猜想.
\end{itemize}
\end{itemize}
*提问: 数学归纳法适用范围是什么?\\
正整数 + 验证正确性. 但不是所有的有关正整数的命题都需要用数学归纳法, 数学归纳法只是众多方法之一.\\
*提问: 我们的步骤应该是怎么样的?\\

\subsection{思考题}
\begin{itemize}
\item[-] 数学归纳法的变形1: 只针对偶自然数或只针对奇自然数.
\item[-] 数学归纳法的变形2: 只针对满足某些条件的自然数.
\end{itemize}

\subsection{作业}
\begin{itemize}
\item 用数学归纳法证明
$$1 + 3 + 5 + \cdots + (2n-1) = n^2.\hspace{3mm}(n \in \mathbb{N}^*)$$
\item (选做)$$S_n = 1 \times 2 + 2 \times 3 + 3 \times 4 + \cdots + n(n+1). \hspace{3mm}(n \in \mathbb{N}^*)$$
$S_1 =  2 = \frac{1}{3} \times 1 \times 2 \times 3,$ \hspace{6mm} $S_2 =  8 = \frac{1}{3} \times 2 \times 3 \times 4,$  \hspace{6mm} $S_3 =  20 = \frac{1}{3} \times 3 \times 4 \times 5,$\\
$S_4 =  40 = \frac{1}{3} \times 4 \times 5 \times 6,$ \hspace{6mm}$S_5 =  70 = \frac{1}{3} \times 5 \times 6 \times 7.$
\begin{enumerate}
\item 猜想$S_n$,
\item 求$f(k) = S_{k+1} - S_k$,
\item 用数学归纳法证明你的猜想.
\end{enumerate}
\end{itemize}

\end{CJK}
\end{document}
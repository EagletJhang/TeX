\documentclass{article}
\usepackage{hyperref}
\usepackage{mathtools}
\usepackage{CJKutf8}
\usepackage{amssymb}
\usepackage{geometry}
\usepackage{enumerate}
\usepackage{multicol} 



\geometry{a4paper} 

\begin{document}
\begin{CJK}{UTF8}{gkai}

\title{数列的极限}
\date{}
\author{张舒悦}
\maketitle

\section{教学目标}
\begin{enumerate}
\item 从数列的变化趋势理解数列极限的概念.
\item 判断一些简单数列的极限.
\end{enumerate}

\section{教学重点}
\begin{enumerate}
\item 数列极限的存在性.
\item 数列极限的趋近方式.
\item 数列极限与数列的项的关系.
\end{enumerate}

\section{教学难点}
\begin{enumerate}
\item 从数列的变化趋势理解数列极限的概念.
\end{enumerate}

\section{教学过程}

\subsection{引入}
\subsubsection{数学史: 关于$\pi$}
我国古代三国时期的数学家刘徽从圆内接正六边形起算, 令边数一倍一倍地增加, 即\\$12, 24, 48, 96, \cdots, 1536, \cdots$, 因而逐个算出正六边形, 正十二边形, 正二十四边形, $\cdots$的周长. 这些多边形的周长逐个得接近圆周. 把圆周分割得细, 误差就越少, 其内接正多边形的周长就越接近圆周. 如此不断地分割下去, 一直到圆周无法再分割为止, 也就是到了圆内接正多边形的边数无限多的时候, 它的周长就与圆周完全一致了。\\
刘徽在这其中运用了极限的思想, 践行了逐步逼近的想法.\\

\subsubsection{计算}
数列${a_n}$, $a_n = \frac{1}{n}$, 完成以下表格(保留三位小数). \\
\begin{tabular}{| c | c | c | c | c | c | c | c | c | c | c | c | c |}
  $n$     & 1 & 2 & 3 & 4 & 5 & 6 & $\cdots$ & 1000 & $\cdots$ & $10^4$ & $\cdots$\\
  $a_n$ &&&&&&&&&&&$\cdots$\\  
\end{tabular}\\
完成后的表格: \\
\begin{tabular}{| c | c | c | c | c | c | c | c | c | c | c | c | c |}
  $n$     & 1 & 2 & 3 & 4 & 5 & 6 & $\cdots$ & 1000 & $\cdots$ & $10^4$ & $\cdots$\\
  $a_n$ & 1 & 0.5 & 0.333 & 0.25 & 0.2 & 0.167 & $\cdots$ & 0.001 & $\cdots$ & 0.000 & $\cdots$\\  
\end{tabular}\\
*提问: $a_n$的变化趋势是怎么样的?\\
随着$n$无限增大, $a_n$和定常数0的距离要多小有多小. 增补表格为: \\
\begin{tabular}{| c | c | c | c | c | c | c | c | c | c | c | c | c |}
  $n$     & 1 & 2 & 3 & 4 & 5 & 6 & $\cdots$ & 1000 & $\cdots$ & $10^4$ & $\cdots$\\
  $a_n$ & 1 & 0.5 & 0.333 & 0.25 & 0.2 & 0.167 & $\cdots$ & 0.001 & $\cdots$ & 0.000 & $\cdots$\\  
  $| a_n - 0 |$ &&&&&&&&&&&$\cdots$
\end{tabular}\\
随着$n$无限增大, $a_n$和定常数0的距离趋近于0. 这也就是说, 随着$n$无限增大, $a_n$无限趋近于定常数0.\\
\newline根据数列的这个关于变化趋势的特点, 我们给出数列极限的定义.\\

\subsection{定义}
一般地, 在$n$无限增大的变化过程中, 如果无穷数列$\{a_n\}$中的$a_n$无限趋近于一个常数A,  那么A叫做数列${a_n}$的极限, 或叫做数列$\{a_n\}$收敛于A, 记作$$\lim_{n \to \infty}{a_n} = A,$$
读作``$n$趋于无穷大时, $a_n$的极限等于A."\\
这个常数A是唯一的.\\
*板书$\lim_{n \to \infty}{\frac{1}{n}} = 0$, 说出读法, 解释含义.\\
\newline
$a_n$无限趋近于A. $\iff$ $a_n$与A的距离要多小有多小.$ \iff $ $|a_n - A|$无限趋近于零.\\
用数学式子表达,\\
*板书$\lim_{n \to \infty}{a_n} = A.\iff \lim_{n \to \infty}{|a_n - A|} = 0.$\\
*板书$\lim_{n \to \infty}{\frac{1}{n}} = 0.\iff \lim_{n \to \infty}{|\frac{1}{n} - 0|} = 0.$

\subsubsection{概念辨析}
判断题: 数列$\underbrace{5, 5, 5, \cdots, 5}_{100000}$的极限是5.\\
有限数列无$n$无限增大的变化过程, 因为无法谈及其相关的变化趋势. 极限定义的大前提是无穷数列.\\
\newline
辨析题:  能否将定义中的``无限趋近"改为``越来越接近"?\\
反例: $a_n = \frac{1}{n}$越来越接近-1. 极限定义中无限趋近常数A的意思是与A的距离要多小有多小.\\

*提问: 是否每一个无穷数列都有极限?
\subsection{极限的存在性}
\subsubsection{$a_n = n^2$}
$a_n$随着$n$的无限增大也无限增大, 不可能无限趋近于一个常数.\\
\subsubsection{$a_n = (-1)^n$}
$a_n$随着$n$的无限增大在1与-1之间摆动, 不可能趋近于一个唯一的常数.\\
\newline
因而面对一个无穷数列, 首先应该判断其是否有极限.
\subsubsection{小试牛刀}
例题1: 判断$a_n = \frac{2n + 1}{n}$极限是否存在, 并说明理由.\\
解:\\先化简: $$a_n = \frac{2n + 1}{n} = 2 + \frac{1}{n}$$\\
观察式子, 其中$\frac{1}{n}$是我们熟悉的. 遂将式子变形为$$a_n - 2 = \frac{1}{n}.$$\\
考虑极限的含义, $$\lim_{n \to \infty}{a_n} = A.\iff \lim_{n \to \infty}{|a_n - A|} = 0.$$\\
再变形为$$|a_n - 2| = \frac{1}{n}\hspace{2mm}.$$\\
利用$\lim_{n \to \infty}{\frac{1}{n}} = 0$, 将上式代入,得$$\lim_{n \to \infty}{|a_n - 2|} = 0.$$


\subsection{趋近于数列极限的基本方式}
例题2: 判断以下数列极限是否存在, 并说明理由.\\
*习题: 分组布置给同学做, 而后和同学一起完成解题.\\
\newline
\textcircled{1} $a_n = (\frac{1}{2})^n.$\\
作图, 点始终在$a_n = 0$的上方; 随着n的无限增大, $a_n$与0的距离要多小有多小.\\
单调递减无限趋近于0, 极限存在, $\lim_{n \to \infty}{a_n} = 0.$\\
\newline
\textcircled{2} $a_n = -(\frac{1}{2})^n.$\\
作图, 点始终在$a_n = 0$的下方; 随着n的无限增大, $a_n$与0的距离要多小有多小.\\
单调递增无限趋近于0, 极限存在, $\lim_{n \to \infty}{a_n} = 0.$\\
\newline
\textcircled{3} $a_n = (-\frac{1}{2})^n.$\\
作图, 点在$a_n = 0$的上下两侧跳跃; 随着n的无限增大, $a_n$与0的距离要多小有多小.\\
跳跃无限趋近于0, 极限存在, $\lim_{n \to \infty}{a_n} = 0.$\\
\newline
\textcircled{1}\textcircled{2}单侧趋近, \textcircled{3}异侧趋近.\\
\newline
\textcircled{4} $a_n = \frac{1}{2}.$\\
作图, 点始终在$a_n =  \frac{1}{2}$上;随着n的无限增大, $a_n$与$\frac{1}{2}$的距离始终为0 .\\
常值趋近于$\frac{1}{2}$, 极限存在, $\lim_{n \to \infty}{a_n} = \frac{1}{2}.$\\
\newline
*提问: 数列极限是否可以等于数列的项?\\
\newpage
\textcircled{5} $a_n = q^n.$\\
当$q > 1$时, 随着n的无限增大, $a_n$正向无限增大, 不可能无限趋近于一个常数.\\
当$q < -1$时, 随着n的无限增大, $a_n$跳跃向正负向无限增大, 不可能无限趋近于一个常数.\\
\newline
综上所述, 当$|q| > 1$或$q = -1$时, 极限不存在; 否则 \\
\begin{displaymath}
\lim_{n \to \infty}{q^n} = \left\{ \begin{array}{ll}
 0 & \textrm{if $|q| < 1$}\\
 q & \textrm{if $q = 1$}\hspace{3mm}.\\
  \end{array} \right.
\end{displaymath}

\subsection{数列的项与极限的关系}
例题3: 判断以下数列极限是否存在, 并说明理由.\\
\begin{displaymath}
a_n = \left\{ \begin{array}{ll}
 2^n & \textrm{if $1 \le n \le 10^6$}\\
 1 & \textrm{if $n > 10^6$}\hspace{3mm}.\\
  \end{array} \right.
\end{displaymath}
*讨论: 极限是否存在, 并说明理由.\\
*预案: 两个极限; 极限不存在.\\
极限与数列的项中前有限项无关, 而与有限项之后的无穷项的变化趋势有关.\\

\subsection{小结}
\begin{itemize}
\item 数列定义: 无穷+无限趋近+定常数.
\item 数列极限不一定存在.
\item 趋近于数列极限的基本方式有单侧, 异侧和常值.
\item 极限与数列的项中前有限项无关, 而与有限项之后的无穷项的*变化趋势*有关.
\end{itemize}

\subsection{作业}
\begin{enumerate}
\item 判断以下数列极限是否存在, 并说明理由.\\1, 0, 1, 0, $\cdots$, $\frac{1}{2}[1+(-1)^{n-1}]$, $\cdots.$
\item 判断以下数列极限是否存在, 并说明理由.\\
\begin{displaymath}
a_n = \left\{ \begin{array}{ll}
 2 & \textrm{if $1 \le n \le 10^{100}$}\\
 \frac{1}{n} & \textrm{if $n > 10^{100}$}\hspace{2mm}.\\
  \end{array} \right.
\end{displaymath} 
\item 满足条件: ``对任意$n \in \mathbb{N}^*$, 都有$a_n < 1$且$ \lim_{n \to \infty}{a_n} = 1."$的无穷数列是否存在? 若存在, 请举例.
\end{enumerate}

\end{CJK}
\end{document}
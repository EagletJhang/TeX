%%
%% Copyright (C) 2012 by Deng Pei <vvcoder@gmail.com>
%%
%% This file may be distributed and/or modified under the conditions of
%% the LaTeX Project Public License, either version 1.3c of this license
%% or (at your option) any later version.  The latest version of this
%% license is in:
%% 
%%    http://www.latex-project.org/lppl.txt
%% 
%% and version 1.3c or later is part of all distributions of LaTeX version
%% 2008/05/04 or later.
%%

%% This is my dissertation use my ecnuthesis class.
%% You can use my dissertation as an example to use
%% ecnuthesis.
%% Any problem, contact me by email:
%%
%% Deng Pei <vvcoder@gmail.com>

\documentclass[a4paper]{ecnuthesis}

\begin{document}

\thesistitle{电子邮件联系人提取工具的设计和实现}{The design and implementation of an email contact extraction tool}
\thesisdegree{学士}
\thesisstudent{邓沛}{10082510246}{08级5班}{软件学院}{软件工程}
\thesisinstructor{罗迒哉}{副教授}
\thesisdate{2011年4月4日}

\thesistitlepage
\thesiscontents

\begin{thesisabstractchinese}
这里是中文摘要。(写你的摘要)
\fontcourier{there is english use Courier fonts}接下来有来一个中文。长行会自动换行,行距为1.5倍普通行距。长行会自动换行,行距为1.5倍普通行距
\end{thesisabstractchinese}

\begin{thesisabstractenglish}
Here is english abstract.(写你的英文摘要)。Long line will auto wrap. Long line will auto wrap. Long line will auto wrap. Long line will auto wrap.
\end{thesisabstractenglish}

\section{绪论}
这里是绪论,包含背景,相关工作,论文组织的介绍。\ref{tab:1}

\subsection{背景}
这里是背景,包含谈中国软件,项目背景,我的项目介绍。

\subsubsection{谈中国软件}
这里我在谈中国软件\urcite{bib1}。

\subsubsection{项目背景\urcite{bib2}}
这里我在谈项目背景。

\subsubsection{我的项目介绍}
这里介绍我的项目。

\subsection{相关工作}
这里是我的做的相关工作。

\subsection{论文组织}
这里时介绍我的论文组织。

\section{系统分析与设计}
这里是系统分析与设计,包括需求分析,概要设计,详细设计。

\subsection{需求分析}
这里是需求分析,包括系统概要和系统流程。

\subsubsection{系统概要}
这里是系统概要。

\subsubsection{系统流程}
这里是系统流程。

\subsection{概要设计}
这里是概要设计,包括业务流程,功能模块介绍。

\subsubsection{业务流程}
这里是业务流程。

\subsubsection{功能模块介绍}
这里是功能模块介绍。

\subsection{详细设计}
这里是详细设计,包括数据库的设计,详细的设计书,界面设计。

\subsubsection{数据库的设计}
这里是数据库的设计。

\subsubsection{详细设计书}
这里是详细设计书。

\subsubsection{界面设计}
这里是界面设计。

\section{系统实现}
这里是系统实现,包括界面,关键算法与数据结构。

\subsection{界面}
这里是界面。

\subsection{关键算法与数据结构}
这里是关键算法与数据结构。

\section{系统测试}
这里是系统测试,包括测试计划,测试用例与测试报告。

\subsection{测试计划}
这里是测试计划,包括单元测试,结合测试,模拟测试,随机测试。
    \begin{table}
        \centering
        \thesistablecaption{tab:1}{索引中的表格1名字}{表格1中文标题}{Table 1 English title}
        \begin{tabular}{l|l|l}
        \hline
        1,1 & 1,2 & 1.3 \\
        \hline
        2,1 & ~ & 2.3 \\
        \hline
        \end{tabular}
    \end{table}
接表格之后的文字。参考\ref{tab:1}

\subsubsection{单元测试}
这里是单元测试。
\begin{center}
    \begin{table}
        \centering
        \thesistablecaption{tab:2}{索引中的表格2名字}{表格2中文标题}{Table 2 English title}
        \begin{tabular}{l|l|l}
        \hline
        1,1 & 1,2 & 1.3 \\
        \hline
        2,1 & ~ & 2.3\ref{tab:1} \\
        \hline
        \end{tabular}
    \end{table}
\end{center}
接表格之\ref{tab:2}后的文字。

\subsubsection{结合测试}
这里是结合测试。

\subsubsection{模拟测试}
这里是模拟测试。

\subsubsection{随机测试}
这里是随机测试。

\subsection{测试用例与测试报告}
这里是测试用例与测试报告。

\section{总结和展望}
这里是总结和展望。

\section{致谢}
到这里来感谢CCTV,感谢挡中央。

\begin{thesisthebibliography}
\bibitem{bib1} 论中国软件的末日。
\bibitem{bib2} 这个引用在标题。
\end{thesisthebibliography}

%\thesistableindex
%\thesisfigureindex
\thesisfigureandtableindex

\begin{thesisappendix}
这里是附录。

\section{LoadMessageFile function}
这里是LoadMessageFile这个函数的介绍附录。

\section{\TeX{} is painful work}
这里\dots 自己想

\end{thesisappendix}

\end{document}

% \iffalse The East China Normal University thesis LaTeX template
%
% Copyright (C) 2012 by Deng Pei <vvcoder@gmail.com>
% -------------------------------------------------------
% 
% This file may be distributed and/or modified under the
% conditions of the LaTeX Project Public License, either
% version 1.3c of this license or (at your option) any later
% version.
% The latest version of this license is in:
%
%    http://www.latex-project.org/lppl.txt
%
% and version 1.3c or later is part of all distributions of LaTeX 
% version 2008/05/04 or later.
%
% \fi
%
% \iffalse
%<*driver>
\ProvidesFile{ecnuthesis.dtx}
%</driver>
%<ecnuthesis>\NeedsTeXFormat{LaTeX2e}
%<ecnuthesis>\ProvidesClass{ecnuthesis}
%<*ecnuthesis>
    [2012/05/26 v3.0 The East China Normal University thesis LaTeX template]
%</ecnuthesis>
%
%<*driver>
\documentclass[10.5pt]{ltxdoc}
%
\usepackage{hyperref}
\usepackage{amsmath}
\usepackage{amsfonts}
\usepackage{fontspec}
\usepackage[slantfont,boldfont]{xeCJK}
%
\setmainfont{Courier New}
\setmonofont{Courier New}
\setCJKmainfont{KaiTi}
\setCJKmonofont{KaiTi}
%
\EnableCrossrefs
\CodelineIndex
\RecordChanges
%
\begin{document}
    \DocInput{ecnuthesis.dtx}
\end{document}
%</driver>
% \fi
%
% \CheckSum{0}
%
% \CharacterTable
%  {Upper-case    \A\B\C\D\E\F\G\H\I\J\K\L\M\N\O\P\Q\R\S\T\U\V\W\X\Y\Z
%   Lower-case    \a\b\c\d\e\f\g\h\i\j\k\l\m\n\o\p\q\r\s\t\u\v\w\x\y\z
%   Digits        \0\1\2\3\4\5\6\7\8\9
%   Exclamation   \!     Double quote  \"     Hash (number) \#
%   Dollar        \$     Percent       \%     Ampersand     \&
%   Acute accent  \'     Left paren    \(     Right paren   \)
%   Asterisk      \*     Plus          \+     Comma         \,
%   Minus         \-     Point         \.     Solidus       \/
%   Colon         \:     Semicolon     \;     Less than     \<
%   Equals        \=     Greater than  \>     Question mark \?
%   Commercial at \@     Left bracket  \[     Backslash     \\
%   Right bracket \]     Circumflex    \^     Underscore    \_
%   Grave accent  \`     Left brace    \{     Vertical bar  \|
%   Right brace   \}     Tilde         \~}
%
% \newcommand{\CTeX}{$\mathbb{C}$\TeX{}}
% \newcommand{\TeXLive}{\TeX{}Live}
%
% \renewcommand{\abstractname}{摘要}
% \renewcommand{\contentsname}{目录}
%
% \IndexPrologue{\section*{索引}%
%     \addcontentsline{toc}{section}{索引}}
% \GlossaryPrologue{\section*{修改}%
%     \addcontentsline{toc}{section}{修改}}
%
% \changes{v1.0}{2012/04/03}{初始化版本。}
% \changes{v1.1}{2012/04/04}{添加了附录,感谢等章节,更加符合华师大的要求。}
% \changes{v1.2}{2012/04/05}{修正了一些小错误,现在基本可以作为正式版本投入使用了。}
% \changes{v1.3}{2012/04/06}{在首页和目录外的页面添加了页眉。添加了用法说明。}
% \changes{v1.4}{2012/04/07}{为了省去大家麻烦,我在默认的zip包中提供预编译的类,同时删除了以前复杂的构建命令,并修改部分说明文档。}
% \changes{v1.5}{2012/04/08}{在目录增加了图表入口,同时修正为中文显示。}
% \changes{v1.6}{2012/04/09}{添加了插图宏包支持。}
% \changes{v1.7}{2012/04/09}{调整了摘要行距为1.5倍普通行距,满足华师大的规定。同时实现了变态的中英文图表标题要求。}
% \changes{v1.8}{2012/04/10}{提供长表环境,另外提供源代码显示环境支持。}
% \changes{v1.9}{2012/04/10}{修正了目录和页码之间的前导点号问题,整体都加上点号。}
% \changes{v2.0}{2012/04/17}{添加并排插图支持。}
% \changes{v2.1}{2012/05/03}{修改初始视图为单页模式。}
% \changes{v2.2}{2012/05/05}{添加合并表格行列支持。}
% \changes{v3.0}{2012/05/26}{根据教务老师的精神,较大改动修正了首页版面,更加和学校保持一致。}
%
% \GetFileInfo{ecnuthesis.dtx}
%
% \DoNotIndex{\newcommand,\newenvironment}
%
% \title{The \textsf{ecnuthesis} class华师大论文\LaTeX{}类}
% \author{邓沛 \\ \texttt{vvcoder@gmail.com}}
% \date{\today}
%
% \maketitle
% \tableofcontents
% \newpage
%
% \section{简介}
%
% 在我编写毕业论文的过程中发现word等工具对管理索引很不方便,而且其对公式
% 的展现效果也不是很好,于是我决定用\LaTeX{}来写论文。为了能方便大家,顺
% 便将自己用的格式整理成了ecnuthesis这个类。希望这个类能简化大家编写论文
% 的任务,让编写论文仅仅成为一件码字的事情,而不需要去考虑其他各种工具特
% 性的细枝末节。
%
% \section{使用}
% 请严格按照我提供的使用方式来使用这个类,不然后果自负。
%
% \subsection{安装}
% 首先你需要安装\TeX{}环境。目前在Windows系统上流行的\TeX{}环境套件很多,
% 你可以有如下选择:
%     \begin{itemize}
%         \item \href{http://www.ctex.org}{\CTeX{}}
%         \item \href{http://www.tug.org/texlive/}{\TeXLive{}}
%     \end{itemize}
% 其中\CTeX{}是专门给中文用户搞的,也是中国人搞的,对Windows支持优秀,但
% 是更新有一些慢。一般是1年1更新(年底更新)。\TeXLive{}是TUG的官方产品,
% 目前对Windows和GNU/Linux支持都很好,推荐这个。当然,除了这2个套件还有很
% 多的套件可以选择,具体自己参考TUG。安装方法具体参考你选择的套件官方说明。
% 安装都很简单,应该不会出什么问题。\par
% 安装完系统,接下来就是安装ecnuthesis类了。本来\LaTeX{}的类安装过程有需要
% 需了解的额外预备知识,但是为了简化大家的工作,我专门编写了一个Make构建文
% 件来帮助大家安装。安装步骤如下:
%     \begin{itemize}
%         \item 下载\href{https://sourceforge.net/projects/ecnuthesis/files/}{ecnuthesis.zip}包。
%               当然,你也可以直接访问本模版的\href{www.ecnuthesis.sf.net}{代码托管}
%               获取最新的版本。建议是直接访问托管库,这样可以获得最新版本,而且以后
%               为了管理方便,可能不会再提供zip包下载。
%         \item 用你喜欢的解压缩工具打开,然后进入\TeX{}环境(命令行提前加入PATH变量)。
%         \item 然后运行make(make是一个工具,你需要自己提前准备,安装方法也比较多,自
%               己选择)。
%     \end{itemize}
% 好了,如果没意外,你已经构建完ecnuthesis类包了。有些复杂,但是在你运行了
% make命令后,会看到很多命令行输出,这些都是编译信息,警告错误信息也会显示
% 在其中。当然,如果你来不及看命令行输出,你可以在make运行完毕后打开ecnuthesis.log
% 里面等同于命令行输出的类容。如果你构建失败,请将这个文件邮寄给我,邮件地
% 址参置顶的联系方式。还有,为了让大多数人免去安装的麻烦,我特地在提供的zip包
% 里面包含了我编译的ecnuthesis.cls,你直接用它也可以,这样你就不用自己编译
% ,也不用去安装make工具,只要有\TeX{}环境就可以了。\par
% 对于make运行后产生的很多文件,除了ecnuthesis.cls是真正的结果文件外,其他
% 的都是垃圾文件,你可以自己删除。让然,你可以运行make clean来删除所有文件
% 包括ecnuthesis.cls文件。ecnuthesis.pdf是ecnuthesis类的说明文件,建议你完
% 整看一遍,这样可以省去你使用ecnuthesis类过程中的诸多麻烦。example.pdf是我
% 使用这个类写的论文例子,你可以参考。\par
% 最后的事情就是将ecnuthesis.cls文件复制到你需要的目录。当然,你也可以复制
% 到/tex/latex/ecnuthesis/目录下(自己创建)。总之,只要\TeX{}系统能找到就
% 行。\par
% 补充:如果你是想自己去make源代码,那在运行make命令后可能会有一些需要输入
% y/n的地方,你全部输入y,然后回车就行,覆盖掉老文件。
% 
% \subsection{功能}
%
% \DescribeMacro{\fontcourier}
% 这个命令用来引用courier字体。我默认设置的是MS的Courier New字体。这个字体
% 对英文显示效果良好,建议你论文中的英文都用这个字体。使用方法如下:\\
% |\fontcourier{Here is english part}|
% 
% \DescribeMacro{\fontkai}
% 这个命令用来引用楷体。楷体对中文显示效果良好,用法如下:\\
% |\fontkai{这里是楷体部分}|
%
% \DescribeMacro{\fontsong}
% 这个命令用来引用宋体。宋体在华师大论文要求的默认中文字体,所以我设置的全
% 文默认字体也是宋体。这里提供这个只是为了完整性:\\
% |\fontsong{这里是宋体部分}|
%
% \DescribeMacro{\fonthei}
% 这个命令用来引用黑体。黑体和粗体类似,都是用于引起读者注意,常用来标题等
% 地方。用法如下:\\
% |\fonthei{这里是黑体部分}|
%
% \DescribeMacro{\urcite}
% 这个命令提供引用功能,比如引用参考文献等。用法:\\
% |专业名词在这里\urcite{reftowhat}| \\
% 这样就会在``专业名词在这里''右上角加一个引用标记。
% reftowhat是你的bib里面写的引用标记。具体看example.tex和
% example.pdf效果。
%
% \DescribeMacro{\thesistitle}
% 设置论文标题,参数3个,第一个参数是中文名字,第二三个参数是英文名字的一二
% 部分,具体怎么断词,你自己决定。使用方法:\\
% |\thesistitle{中文标题}{Chinese}{title}|
%
% \DescribeMacro{\thesisdate}
% 设置论文日期。用法:\\
% |\thesisdate{1111年11月11日}|
%
% \DescribeMacro{\thesisdegree}
% 设置论文申请的学位,一般是学士/硕士/博士,使用方法:\\
% |\thesisdegree{学士}|
%
% \DescribeMacro{\thesisstudent}
% 设置论文的学生信息,有5个参数,第一个参数是名字,第二个参数是学号,第三个参
% 数是班级,第四个参数是学院,第五个参数是专业。用法如下:\\
% |\thesisstudent{张三}{101010101}{01级01班}{计算机学院}{计算机专业}|
%
% \DescribeMacro{\thesisinstructor}
% 设置导师信息,有2个参数,第一个参数是导师姓名,第二个参数是导师职称。
% 用法如下:\\
% |\thesisinstructor{阿三}{叫授}|
%
% \DescribeMacro{\thesistitlepage}
% 创建标题页。当然你设置完前面的标题页需要显示的信息后,用这个命令来真正创建标题页。
% 用法如下:\\
% |\thesistitlepage| \\
% 不需要参数,就这样就ok了。
%
% \DescribeMacro{\thesiscontents}
% 创建目录页。当你需要创建目录的时候用这个命令,立即自动生成目录。用法很简单,如下:\\
% |\thesiscontents| \\
% 所有目录的标号以及索引和连接都自动创建完了。不需要你去``选择更新域''。
%
% \DescribeMacro{\thesistablecaption}
% 创建表格的标题,这个命令有4个参数,第一个参数是引用标记,第二参数是短标题,第三个
% 参数是中文标题,第四个参数是英文标题。
%
% \DescribeMacro{\thesistableindex}
% 如果你想在论文的某个地方,比如结束的时候,列举一个论文中引用到的所有的\textbf{表}
% 的列表,你运行这个命令会自动生成让你满意的结果。用法如下:\\
% |\thesistableindex|
%
% \DescribeMacro{\thesisfigurecaption}
% 创建插图的标题,和|\thesistablecaption|命令参数类似。
%
% \DescribeMacro{\thesisfigureindex}
% 和|\thesistableindex|效果类是,只是生成的是所有引用到的\textbf{图}的列表,用法如下:\\
% |\thesisfigureindex|
%
% \DescribeMacro{\thesisfigureandtableindex}
% 和|\thesistableindex|,|\thesisfigureindex|效果类似,只是将插图和表格列表放在一页上。
% 用法如下:\\
% |\thesistableandfigureindex|
%
% \DescribeEnv{thesisabstractchinese}
% 中文摘要环境。让你要写中文摘要的时候需要用到这个环境,用法如下:
% |\begin{thesisabstractchinese}| \\
% 这是是中文摘要内容 \\
% |\end{thesisabstractchinese}| \\
% 环境用起来稍微比命令复杂一些,但是也很简单。
%
% \DescribeEnv{thesisabstractenglish}
% 中文摘要环境。让你要写中文摘要的时候需要用到这个环境,用法如下:
% |\begin{thesisabstractenglish}| \\
% Here is english abstract \\
% |\end{thesisabstractenglish}|
%
% \DescribeEnv{thesisthebibliography}
% 参考文献环境。用法请参考example.exe,效果请看example.pdf。这里不赘述。
%
% \DescribeEnv{thesisappendix}
% 附录环境。用法请参考example.exe,效果请看example.pdf。这里不赘述。
%
% \StopEventually{\PrintChanges\PrintIndex}
%
% \section{实现}
%
% 为了简化,这里的实现都来自于我为这个类编写的代码,对于感兴趣的朋友,
% 可以作为参考。不过对于仅仅想使用这个类的功能,那么了解实现是没有什么
% 意义的。
%
% \subsection{基本设置}
%
%    \begin{macrocode}
% 使用默认的article类作为基础来创建类。
\LoadClassWithOptions{article}
%    \end{macrocode}
%
%    \begin{macrocode}
% 导入一些常用的宏包,方便你在写论文的过程中使用。 
\RequirePackage{xcolor}                                     % 色彩支持宏包
\RequirePackage{hyperref}                                   % 创建超链接和目录宏包
\RequirePackage{amsmath}                                    % 数学符号宏包
\RequirePackage{amsfonts}                                   % 数学字体宏包
\RequirePackage{fontspec}                                   % 设置字体宏包
\RequirePackage[slantfont,boldfont]{xeCJK}                  % XeLaTeX的CJK支持宏包
\RequirePackage[toc,page,title,titletoc,header]{appendix}   % 附录支持宏包
\RequirePackage{fancyhdr}                                   % 页眉页脚支持宏包
\RequirePackage{graphicx}                                   % 插图支持
\RequirePackage{setspace}                                   % 行距支持宏包
\RequirePackage{ccaption}                                   % 中英文标题支持宏包
\RequirePackage{longtable}                                  % 长表格环境支持宏包
\RequirePackage{listings}                                   % 源代码显示支持宏包
\RequirePackage[titles,subfigure]{tocloft}                  % 目录和页码点号支持宏包
\RequirePackage{subfigure}                                  % 并排插图支持宏包
\RequirePackage{multirow}                                   % 合并表格行列支持
\RequirePackage{ulem}                                       % 删除线支持
\RequirePackage{geometry}                                   % 页面边距调整支持
\RequirePackage{calc}                                       % 计算支持
%    \end{macrocode}
%
%    \begin{macrocode}
% 对超链接/PDF显示效果的设置。
\hypersetup{%
    pdfencoding=auto,			% 自动确定编码
    pdfpagelayout={SinglePage}, % 初始单页视图
    bookmarksnumbered={true},	% 书签编号
    bookmarksopen={false},		% 自动展开书签
    pdfborder={0 0 0}}			% 边框
%    \end{macrocode}
%
%    \begin{macrocode}
% 设置目录和页码之间的点号。
\renewcommand{\cftsecleader}{\cftdotfill{\cftdotsep}}
%    \end{macrocode}
%
%    \begin{macrocode}
% 设置中文段首,LaTeX原来默认的是2个M宽度。后来随着发展,变成了2个ccwd,而
% ccwd一般会考虑CJKglue,但是问题出在parindent不会随着字体大小编码而自动扩展,
% 所以我这里把parindent改成了2.5个M,这样在五号,也就是10.5pt的时候显示起来更
% 美观。你可以根据实际使用的字体大小来修改parindent。
\setlength{\parindent}{2.5em}
%    \end{macrocode}
%
% \begin{macro}{\fontcourier}
% courier是苹果设计的衬线字体,适合长期阅读而不疲劳,也是非常美观的英文字体。
%    \begin{macrocode}
\newfontfamily\fontcourier{Courier New} % courier,我觉得最漂亮的英文字体
%    \end{macrocode}
% \end{macro}
%
% \begin{macro}{\fontkai}
% 楷体是中文显示很均衡的字体,看起来有力道也非常柔和,适合长时间阅读。
%    \begin{macrocode}
\newfontfamily\fontkai{KaiTi} % 楷体,我觉得最漂亮的中文字体
%    \end{macrocode}
% \end{macro}
%
% \begin{macro}{\fontsong}
% 宋体是使用非常广泛的中文字体,为了遵循官方要求本论文模板类也使用宋体作为默认
% 字体,但是个人认为其非常不美观,尤其是字体小于12pt的时候,完全不能体现汉字的
% 美感,所以如果你需要,你可用前面的楷体。
%    \begin{macrocode}
\newfontfamily\fontsong{NSimSun} % 宋体,很难看但是很常用的中文字体
%    \end{macrocode}
% \end{macro}
%
% \begin{macro}{\fonthei}
% 黑体是用来突出醒目强调文字用的,比如章节标题等等。
%    \begin{macrocode}
\newfontfamily\fonthei{SimHei} % 黑体
%    \end{macrocode}
% \end{macro}
%
%    \begin{macrocode}
% 全文字体默认设置
\setmainfont{Times New Roman}   % 默认字体
\setmonofont{Times New Roman}   % 默认等宽字体
\setsansfont{Arial}             % 英文无衬线字体
\setCJKmainfont{NSimSun}        % CJK默认字体
\setCJKmonofont{NSimSun}        % CJK等宽字体
%    \end{macrocode}
%
%   \begin{macrocode}
% 设置页边距
\geometry{left=2.5cm,right=2.5cm,top=2.5cm,bottom=2.5cm}
%   \end{macrocode}
%
%    \begin{macrocode}
% 设置华师大要求的页眉
\pagestyle{fancy}
\fancyhead[L]{{\color{gray}{华东师范大学\@deg 学位论文}}}
\fancyhead[R]{{\color{gray}{\@titzh}}}
\renewcommand{\headrule}{{\color{gray}{\hrule}}}
%    \end{macrocode}
%
%    \begin{macrocode}
% 设置图表标题和上下文之间的间隔,保持美观。
%\setlength{\abovecaptionskip}{2\baselineskip}
\setlength{\abovelegendskip}{\baselineskip}
%\setlength{\belowcaptionskip}{2\baselineskip}
\setlength{\belowlegendskip}{\baselineskip}
%    \end{macrocode}
%
% \begin{macro}{\urcite}
% 右上角标注引用命令,第一个参数是引用标记。一般在常见的标注引用是使用[1]
% 这种样式,但是为了遵循华师大的标准要求,我提供了urcite,即up-right cite。
%    \begin{macrocode}
\newcommand{\urcite}[1]{%
    \textsuperscript{%
        \textsuperscript{\cite{#1}}}}%
%    \end{macrocode}
% \end{macro}
%
% \begin{macro}{\thesistitle}
% 设置论文标题页标题的命令,第一个参数是中文标题,第二个参数是英文标题第一部分,
% 第三个参数是英文标题第二部分。这样设计的考虑是有时候英文标题可能过长而导致显示
% 不美观。具体英文第一行和第二行怎么显示,断行由你来决定,你只需要传入两部分参数
% 即可。
%    \begin{macrocode}
\newcommand{\thesistitle}[2]{% 标题页标题
    \def\@titzh{#1}     % 中文标题
    \def\@titen{#2}}    % 英文标题
%    \end{macrocode}
% \end{macro}
%
% \begin{macro}{\thesisdate}
% 设置论文标题页日期的命令,第一个参数是日期。
%    \begin{macrocode}
\newcommand{\thesisdate}[1]{% 标题页日期
    \def\@dat{#1}}      % 标题页日期
%    \end{macrocode}
% \end{macro}
%
% \begin{macro}{\thesisdegree}
% 设置论文标题页学位的命令,第一个参数是学位(一般是学士/硕士/博士)。
%    \begin{macrocode}
\newcommand{\thesisdegree}[1]{% 标题页学位
    \def\@deg{#1}}      % 标题页学位
%    \end{macrocode}
% \end{macro}
%
% \begin{macro}{\thesisstudent}
% 设置论文标题页学生的命令,第一个参数是名字,第二个参数是学号,第三个参数是班级,
% 第四个参数是学院,第五个参数是专业。
%    \begin{macrocode}
\newcommand{\thesisstudent}[5]{% 标题页学生信息
    \def\@stuname{#1}   % 学生姓名
    \def\@stuid{#2}     % 学生学号
    \def\@stucls{#3}    % 学生班级
    \def\@stuins{#4}    % 学生学院
    \def\@stumaj{#5}}   % 学生专业
%    \end{macrocode}
% \end{macro}
%
% \begin{macro}{\thesisinstructor}
% 设置论文标题页导师的命令,第一个参数是名字,第二个参数是职称。
%    \begin{macrocode}
\newcommand{\thesisinstructor}[2]{% 标题页导师信息
    \def\@insname{#1}   % 导师姓名
    \def\@inslevel{#2}} % 导师职称
%    \end{macrocode}
% \end{macro}
%
% \begin{macro}{\thesistitlepage}
% 创建标题页的命令。
%    \begin{macrocode}
\newcommand{\thesistitlepage}{
    \begin{titlepage}
        \fbox{\begin{minipage}[s]{\widthof{华东师范大学\@stuins}}
            华东师范大学\@stuins\@stumaj\@deg 学位论文
            \end{minipage}}
        \begin{center}
            \vspace{30mm}
            \huge
            \textbf{\@titzh}\par
            \textbf{\@titen}\par
            \vspace{40mm}
            \Large
            \textbf{姓 \qquad 名} \underline{\hfill \makebox[80mm]{\@stuname} \hfill} \par
            \vspace{5mm}
            \textbf{学 \qquad 号} \underline{\hfill \makebox[80mm]{\@stuid} \hfill} \par
            \vspace{5mm}
            \textbf{班 \qquad 级} \underline{\hfill \makebox[80mm]{\@stucls} \hfill} \par
            \vspace{5mm}
            \textbf{导 \qquad 师} \underline{\hfill \makebox[25mm]{\@insname} \hfill} \quad \textbf{职 \qquad 称} \underline{\hfill \makebox[25mm]{\@inslevel} \hfill} \par
            \vspace{5mm}
            \textbf{日 \qquad 期} \underline{\hfill \makebox[80mm]{\@dat} \hfill} \par
        \end{center}
    \end{titlepage}}
%    \end{macrocode}
% \end{macro}
%
% \begin{macro}{\thesiscontents}
% 创建目录的命令。
%    \begin{macrocode}
\newcommand{\thesiscontents}{%
    \renewcommand{\contentsname}{\centerline{\huge\textbf{目~~~~录}}}%
    \thispagestyle{empty}%
    \tableofcontents}%
%    \end{macrocode}
% \end{macro}
%
% \begin{environment}{thesisabstractchinese}
% 创建中文摘要的环境。
%    \begin{macrocode}
\newenvironment{thesisabstractchinese}{%
    \newpage%
    \renewcommand{\abstractname}{\centerline{\huge\textbf{摘~~~~要}}}%
    \begin{abstract}\begin{spacing}{1.5}}{\end{spacing}\end{abstract}}%
%    \end{macrocode}
% \end{environment}
%
% \begin{environment}{thesisabstractenglish}
% 创建英文摘要的环境。
%    \begin{macrocode}
\newenvironment{thesisabstractenglish}{%
    \newpage%
    \renewcommand{\abstractname}{\centerline{\huge\textbf{Abstract}}}%
    \begin{abstract}\begin{spacing}{1.5}}{\end{spacing}\end{abstract}\newpage}%
%    \end{macrocode}
% \end{environment}
%
% \begin{macro}{\thesistablecaption}
%    \begin{macrocode}
% 创建表格的中英文标题
\renewcommand{\tablename}{表}
\newcommand{\thesistablecaption}[4]{\bicaption[#1]{#2}{#3}{Table}{#4}\vspace{\baselineskip}}
%    \end{macrocode}
% \end{macro}
%
% \begin{macro}{\thesistableindex}
% 创建表格索引,新建页。
%    \begin{macrocode}
\newcommand{\thesistableindex}{%
    \newpage%
    \renewcommand{\listtablename}{\centerline{\huge\textbf{表~~格~~列~~表}}}
    \listoftables%
    \addcontentsline{toc}{section}{表格列表}}%
%    \end{macrocode}
% \end{macro}
%
% \begin{macro}{\thesisfigurecaption}
%    \begin{macrocode}
% 创建插图的中英文标题
\renewcommand{\figurename}{图}
\newcommand{\thesisfigurecaption}[4]{\vspace{\baselineskip}\bicaption[#1]{#2}{#3}{Figure}{#4}}
%    \end{macrocode}
% \end{macro}
%
% \begin{macro}{\thesisfigureindex}
% 创建插图索引,新建页。
%    \begin{macrocode}
\newcommand{\thesisfigureindex}{%
    \newpage%
    \renewcommand{\listfigurename}{\centerline{\huge\textbf{插~~图~~列~~表}}}
    \listoffigures%
    \addcontentsline{toc}{section}{插图列表}}%
%    \end{macrocode}
% \end{macro}
%
% \begin{macro}{\thesisfigureandtableindex}
% 创建表格和插图表列,并放在同一页。
%    \begin{macrocode}
\newcommand{\thesisfigureandtableindex}{%
    \newpage%
    \renewcommand{\listfigurename}{\centerline{\huge\textbf{插~~图~~列~~表}}}
    \listoffigures%
    \renewcommand{\listtablename}{\centerline{\huge\textbf{表~~格~~列~~表}}}
    \listoftables%
    \addcontentsline{toc}{section}{图表列表}}%
%    \end{macrocode}
% \end{macro}
%
% \begin{environment}{thesisthebibliography}
% 创建参考文献
%    \begin{macrocode}
\newenvironment{thesisthebibliography}{%
    \newpage%
    \renewcommand{\refname}{\centerline{\huge\textbf{参~~考~~文~~献}}}%
    \begin{thebibliography}{64}}%
    {\end{thebibliography}%
    %\addcontentsline{toc}{section}{参考文献 \dotfill}}% 可能会导致目录溢出,所以不用这个。
    \addcontentsline{toc}{section}{参考文献}}%
%    \end{macrocode}
% \end{environment}
%
% \begin{environment}{thesisappendix}
% 创建附录
%    \begin{macrocode}
\newenvironment{thesisappendix}{%
    \newpage%
    \renewcommand{\appendixname}{}%
    %\renewcommand{\appendixtocname}{附录 \dotfill}% 可能会导致目录溢出,所以不用这个。
    \renewcommand{\appendixtocname}{附录}%
    \renewcommand{\appendixpagename}{\centerline{\huge\textbf{附~~~~录}}}%
    \begin{appendices}}%
    {\end{appendices}}%
%    \end{macrocode}
% \end{environment}
%
% \Finale
\endinput

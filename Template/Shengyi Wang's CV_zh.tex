\documentclass[11pt, a4paper]{moderncv}
\moderncvtheme[blue]{classic}
\usepackage[utf8]{inputenc}
\usepackage{CJKutf8}
\usepackage[scale=0.85]{geometry}
\setlength{\hintscolumnwidth}{3cm}
\firstname{王}
\familyname{盛颐}
\mobile{13810247628}
\email{txyyss@gmail.com}
\extrainfo{\upshape\url{https://sites.google.com/site/txyyss/}}

\begin{document}
\begin{CJK}{UTF8}{gkai}       
\maketitle

\section{教育背景}
\cventry{2007 -- 2010}{应用数学硕士}{北京大学}{中国北京}{}{}
\cventry{2003 -- 2007}{基础数学学士}{北京大学}{中国北京}{}{}

\section{工作经历}
\cventry{2010 年 7 月 --}{软件工程师}{腾讯科技(北京)有限公司}{中国北京}{}{先在 \url{http://wenwen.soso.com} 然后是 \url{https://tg.qq.com}}

\cventry{2007 年 3 月 -- 2007 年 12 月}{实习生}{IBM 中国研究院}{中国北京}{}{在信息可视化组工作, 期间申请专利一项。}

\section{学术研究}
\cventry{2010 年 8 月}{Stack Bound Inference for Abstract Java Bytecode}{TASE 2010}{Taipei, Taiwan}{}{}{}

\cventry{2009 年 7 月}{A Tool For Estimating Memory Usage}{TASE 2009}{Tianjin, China}{}{}{}

\section{软件项目}
\cvline{Grid Maze}{iPad 版本的文字迷宫,可将任意文字或图像转换成迷宫的解答路径\newline
\url{http://goo.gl/afNJt}}
\cvline{几何剖分游戏}{一个关于几何剖分的软件原型 \url{http://goo.gl/hFWND}}
\cvline{不可能图形}{不可能图形的自动绘制生成 \url{http://goo.gl/3tBk2}}
\cvline{文字迷宫}{可自定义求解路径为文字的迷宫 \url{http://goo.gl/7gybC}}
\cvline{内存分析}{通过静态分析估算程序内存占用大小的程序}
\cvline{Ferret}{用 Java 实现的一个完整的计算机代数系统}

\section{获得荣誉}
\cventry{2007 年 3 月}{一等奖}{2007 年美国数学建模比赛}{美国}{}{}
\cventry{2006 年 11 月}{社会工作奖}{北京大学}{中国北京}{}{}
\cventry{2006 年 3 月}{一等奖}{2006 年美国数学建模比赛}{美国}{}{}

\section{计算机技能}
\cvline{程序设计语言}{Common Lisp, C, C++, Haskell, Java, Objective-C, PostScript}
\cvline{熟悉的工具}{Mathematica, \LaTeX, emacs, Qt, boost}

\clearpage\end{CJK}
\end{document}
